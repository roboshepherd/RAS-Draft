%% Template article for Elsevier's document class `elsarticle'
%% with numbered style bibliographic references
%% SP 2008/03/01
%%
%%
%%
%% $Id: elsarticle-template-num.tex 4 2009-10-24 08:22:58Z rishi $
%%
%%
\documentclass[preprint,12pt]{./elsart/elsarticle}
%% Use the option review to obtain double line spacing
%% \documentclass[preprint,review,12pt]{elsarticle}
%% Use the options 1p,twocolumn; 3p; 3p,twocolumn; 5p; or 5p,twocolumn
%% for a journal layout:
%% \documentclass[final,1p,times]{elsarticle}
%% \documentclass[final,1p,times,twocolumn]{elsarticle}
%% \documentclass[final,3p,times]{elsarticle}
%% \documentclass[final,3p,times,twocolumn]{elsarticle}
%% \documentclass[final,5p,times]{elsarticle}
%% \documentclass[final,5p,times,twocolumn]{elsarticle}

%% if you use PostScript figures in your article
%% use the graphics package for simple commands
%% \usepackage{graphics}
%% or use the graphicx package for more complicated commands
\usepackage{graphicx}
%% or use the epsfig package if you prefer to use the old commands
%% \usepackage{epsfig}

%% The amssymb package provides various useful mathematical symbols
\usepackage{amssymb}
%% The amsthm package provides extended theorem environments
%% \usepackage{amsthm}

%% The lineno packages adds line numbers. Start line numbering with
%% \begin{linenumbers}, end it with \end{linenumbers}. Or switch it on
%% for the whole article with \linenumbers after \end{frontmatter}.
%% \usepackage{lineno}

%% natbib.sty is loaded by default. However, natbib options can be
%% provided with \biboptions{...} command. Following options are
%% valid:

%%   round  -  round parentheses are used (default)
%%   square -  square brackets are used   [option]
%%   curly  -  curly braces are used      {option}
%%   angle  -  angle brackets are used    <option>
%%   semicolon  -  multiple citations separated by semi-colon
%%   colon  - same as semicolon, an earlier confusion
%%   comma  -  separated by comma
%%   numbers-  selects numerical citations
%%   super  -  numerical citations as superscripts
%%   sort   -  sorts multiple citations according to order in ref. list
%%   sort&compress   -  like sort, but also compresses numerical citations
%%   compress - compresses without sorting
%%
%% \biboptions{comma,round}
% \biboptions{}
\journal{Robotics and Autonomous System}
\begin{document}
%%
\begin{frontmatter}
%%
%% Title, authors and addresses
%%
%% use the tnoteref command within \title for footnotes;
%% use the tnotetext command for the associated footnote;
%% use the fnref command within \author or \address for footnotes;
%% use the fntext command for the associated footnote;
%% use the corref command within \author for corresponding author footnotes;
%% use the cortext command for the associated footnote;
%% use the ead command for the email address,
%% and the form \ead[url] for the home page:
%%
%% \title{Title\tnoteref{label1}}
%% \tnotetext[label1]{}
%% \author{Name\corref{cor1}\fnref{label2}}
%% \ead{email address}
%% \ead[url]{home page}
%% \fntext[label2]{}
%% \cortext[cor1]{}
%% \address{Address\fnref{label3}}
%% \fntext[label3]{}
%%
\title{Self-regulated Multi-robot Task~Allocation:\\ A Taxonomy and Comparison of Centralized and Local Communication Strategies}
%%
%% use optional labels to link authors explicitly to addresses:
\author[label1]{Md Omar Faruque Sarker and Torbj{\o}rn S. Dahl}
\address[label1]{Cognitive Robotics Research Centre\\
Newport Business School,
Allt-yr-yn Campus\\ %Allt-yr-yn Avenue\\
Newport, NP20 5DA,
United Kingdom.\\
Mdomarfaruque.Sarker \vline Torbjorn.Dahl@newport.ac.uk}
%\address[label1]{<address>}
%
%\author{}
%
%\address{}
%
\begin{abstract}
This paper proposes to solve the MRTA problem using a set of previously published generic rules for division of labour derived from the observation of ant, human and robotic social systems. The concrete form of these rules, the \textit{attractive filed model} (AFM), provides sufficient abstraction to local communication and sensing which is uncommon in existing MRTA solutions. We have validated the effectiveness of AFM to address MRTA  using two bio-inspired communication and sensing strategies: ``global sensing - no communication'' and ``local sensing - local communication''. The former is realized using a centralized communication system and the latter is emulated under a peer-to-peer local communication scheme. They are applied in a  manufacturing shop-floor scenario using 16 e-puck robots. A flexible multi-robot control architecture, \textit{hybrid event-driven architecture on D-Bus}, has been outlined which uses the state-of-the-art D-Bus interprocess communication.  Based-on the organization of task-allocation, communication and interaction among robots, a  novel taxonomy of MRTA solutions has been proposed to remove the ambiguities found in existing MRTA solutions. Besides, a set of domain-independent metrics, e.g., plasticity, task-specialization and energy usage, has been formalized to compare the performances of the above two strategies.
\end{abstract}
%%
\begin{keyword}
multi-robot system \sep multi-robot task allocation
%% keywords here, in the form: keyword \sep keyword
%% MSC codes here, in the form: \MSC code \sep code
%% or \MSC[2008] code \sep code (2000 is the default)
\end{keyword}
%%
\end{frontmatter}
%%
%%
%% Start line numbering here if you want
%%
% \linenumbers

%% main text
\section{Introduction}
\label{sec:intro}
%%
\section{The Attractive Field Model}
\label{sec:afm}
%%
\section{Related work}
\label{sec:bg}
%%
\section{AFM based task-allocation solution}
\label{sec:mrta}
%%
\section{Experiments}
\label{sec:expt}
%%
\section{Results}
\label{sec:res}
%%
\section{Discussions}
\label{sec:discuss}
%%
\section{Conclusions}
\label{sec:conc}
%% The Appendices part is started with the command \appendix;
%% appendix sections are then done as normal sections
%% \appendix
%% \section{}
%% \label{}
%% References
%%
%% Following citation commands can be used in the body text:
%% Usage of \cite is as follows:
%%   \cite{key}         ==>>  [#]
%%   \cite[chap. 2]{key} ==>> [#, chap. 2]
%%

%% References with bibTeX database:
\bibliographystyle{elsarticle-num}
\bibliography{<your-bib-database>}
%% Authors are advised to submit their bibtex database files. They are
%% requested to list a bibtex style file in the manuscript if they do
%% not want to use elsarticle-num.bst.
%% References without bibTeX database:
% \begin{thebibliography}{00}
%% \bibitem must have the following form:
%%   \bibitem{key}...
%%
% \bibitem{}
% \end{thebibliography}
\end{document}
%%
%% End of file `elsarticle-template-num.tex'.